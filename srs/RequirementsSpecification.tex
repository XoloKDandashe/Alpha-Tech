\documentclass[english]{article}

\usepackage{graphicx}
\usepackage{grffile}
\usepackage{babel}
\usepackage{parskip}
\textwidth = 426pt
\oddsidemargin = 17pt


\title{Cos301 : Software Requirements Specification\\
	for the ...... System\\
	}
\date{\today}
\graphicspath{{Pictures/}}

\begin{document}
	\maketitle
	\begin{figure}[!t]
		\includegraphics{up_logo.png}
	\end{figure}
	\begin{minipage}{0.4\textwidth}
		\begin{flushleft} \large
			\textbf{NAMES:}\\[0.4cm]
			Mufamadi {Khodani} \\
			Setoaba {Phuti} \\ 
		\end{flushleft}
	\end{minipage}
	\begin{minipage}{0.4\textwidth}
		\begin{flushright} \large
			\textbf{STUDENT NUMBER:} \\[0.4cm]
			u14197520 \\
			u13032616 \\
		\end{flushright}
	\end{minipage}


	
	\pagenumbering{gobble}
	\newpage

	\tableofcontents
	\newpage

	\pagenumbering{arabic}
	

	\section{Introduction}
			

		\subsection{Purpose}
			The purpose of this document is to outline the requirements and design specifications of the SmartCard system. This document provides a detailed explanation of system constraints, interfaces and interactions with other external applications. This document serves as a reference for developing the first version of the system for the development team and will also provide a detailed description of the system for clients and any other interested party.


		\subsection{Scope}


		\subsection{Definition, Acronyms, and Abbreviations}
				This section of the SRS contains definitions, acronyms and abbreviations for the terminology used to describe our system throughout this document.
				\\
				\\
				\begin{tabular}{ |p{3cm}|p{9cm}|  }
				\hline
				\textbf{Term} & \textbf{Definition}\\
				\hline
				User & An actor that interacts with the social media platform\\
				\hline
				Administrator & An actor that is given specific permission for managing and controlling the system\\
				\hline
				\end{tabular}

		\subsection{References}
			[1] IEEE Software Engineering Standards Committee, "IEEE Std 830-1998, IEEE Recommended Practice for Software Requirements Specifications", October 20,1998

		\subsection{Overview}
				\begin{tabular}{ |p{3cm}||p{11cm}|  }

				\end{tabular}
	\newpage
	\section{Overall Description}
		
		\subsection{Product Perspective}
			
				\subsubsection{System Interfaces}
		

				\subsubsection{User Interfaces}
						

				\subsubsection{Hardware Interfaces}
				   
				\subsubsection{Software Interfaces}
			
				\subsubsection{Communications Interfaces}
			
				\subsubsection{Memory}

				\subsubsection{Operations}

				\subsection{Product Functions}

				\subsection{User Characteristics}
				
			
				\subsection{Constraints}
            

				\subsection{Assumptions and Dependencies}
				\subsubsection{Assumptions}


	
				\begin{large}
				\subsubsection{Dependencies}
				\end{large}
		
		\newpage

	\section{Specific Requirements}
				\subsection{External Interface Requirements}
				\subsection{Functional Requirements}
				\subsection{Performance Requirements}
				\subsection{Design Constraints}
				\paragraph\indent
				This application is constrained in two main areas, namely hardware and software. We will look at these two areas separately.
				 \subsubsection{Hardware}
				
				 \subsubsection{Software}
			
				 \subsubsection{Other}
				
					
									\newpage
				\subsection{Software System attributes}
				
				\subsubsection{Reliability}
    			
    				
				\subsubsection{Portability}
 
    				
				\subsubsection{Robustness}
    		
    				
				\subsubsection{Security}
    			
    				
				\subsubsection{Reusability}
	
				    
				\subsubsection{Efficiency}
				  
				\subsubsection{Availability}
				 
	
				\subsubsection{Interoperability}
				

				\subsection{Other Requirements}
				

	
\end{document}
