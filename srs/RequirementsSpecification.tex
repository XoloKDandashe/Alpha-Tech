\documentclass[english]{article}

\usepackage{graphicx}
\usepackage{grffile}
\usepackage{babel}
\usepackage{parskip}
\textwidth = 426pt
\oddsidemargin = 17pt


\title{Cos301 : Software Requirements Specification\\
	for the ...... System\\
	}
\date{\today}
\graphicspath{{Pictures/}}

\begin{document}
	\maketitle
	\begin{figure}[!t]
		\includegraphics{up_logo.png}
	\end{figure}
	\begin{minipage}{0.4\textwidth}
		\begin{flushleft} \large
			\textbf{NAMES:}\\[0.4cm]
			Mufamadi {Khodani}
		\end{flushleft}
	\end{minipage}
	\begin{minipage}{0.4\textwidth}
		\begin{flushright} \large
			\textbf{STUDENT NUMBER:} \\[0.4cm]
			u14197520
		\end{flushright}
	\end{minipage}


	
	\pagenumbering{gobble}
	\newpage

	\tableofcontents
	\newpage

	\pagenumbering{arabic}
	

	\section{Introduction}
			

		\subsection{Purpose}
			

		\subsection{Scope}


		\subsection{Definition, Acronyms, and Abbreviations}
				This section of the SRS contains definitions, acronyms and abbreviations for the terminology used to describe our system throughout this document.
				\\
				\\
				\begin{tabular}{ |p{3cm}|p{9cm}|  }
				\hline
				\textbf{Term} & \textbf{Definition}\\
				\hline
				User & An actor that interacts with the social media platform\\
				\hline
				Administrator & An actor that is given specific permission for managing and controlling the system\\
				\hline
				\end{tabular}

		\subsection{References}
			[1] IEEE Software Engineering Standards Committee, "IEEE Std 830-1998, IEEE Recommended Practice for Software Requirements Specifications", October 20,1998

		\subsection{Overview}
				\begin{tabular}{ |p{3cm}||p{11cm}|  }

				\end{tabular}
	\newpage
	\section{Overall Description}
		
		\subsection{Product Perspective}
			
				\subsubsection{System Interfaces}
		

				\subsubsection{User Interfaces}
						

				\subsubsection{Hardware Interfaces}
				   
				\subsubsection{Software Interfaces}
			
				\subsubsection{Communications Interfaces}
			
				\subsubsection{Memory}

				\subsubsection{Operations}

				\subsection{Product Functions}

				\subsection{User Characteristics}
				
			
				\subsection{Constraints}
            

				\subsection{Assumptions and Dependencies}
				\subsubsection{Assumptions}


	
				\begin{large}
				\subsubsection{Dependencies}
				\end{large}
		
		\newpage

	\section{Specific Requirements}
				\subsection{External Interface Requirements}
						Since the prototype will  be developed for Android systems and later on expanded to iOS ,this part of the specification will 		                                                primarily assume the Android OS is the operating system for the system.
						\subsubsection{Interfaces} 
					         The app will require system privileges to make use of certain features and information provided by the Android OS.
						 \begin{itemize}
							      \item \textbf{NFC}\\
					        The system will be dependant on the device's NFC system services to be able to read the nfc smart card and also send                              business cards via NFC to another device.Access to this feature forms the backbone of the app as most of the subsystems are dependant on it.
						\end{itemize}
						\subsubsection{User Interfaces}
						    There will be a unified user interface which can be broken down into 4 subsystems all of which will be displayed 		           simultaneously. The 5th subsystem (Searching) will tie into the navigational subsystem \\
                                                  \begin{itemize}
					        \item \textbf{Send Card}\\

						\item \textbf{Recieve Card}\\

						\item \textbf{Update Details}\\

						\item \textbf{Manage Cards}\\


                                                \item \textbf{Searching}\\
						This will be a minor add-on to the Navigational Subsystem which will allow users to search for business cards they have saved on their device.
                                                  \end{itemize}
                                              \subsubsection{System Interfaces}
							\begin{itemize}
    						    \item \textbf{An android phone } \\
    						    Android devices are the most commonly used devices, so development will initially focus on these devices. There are android devices with many varying specifications, but we will focus on newer models in order to simplify the prototype. \\
    
    						\end{itemize}
                                               \subsubsection{Software Interfaces}
        					\begin{itemize}
        					    \item \textbf{Android NFC API}\\
        					    This would be our tool of choice as it covers the basis for sending and receiving data through NFC ,which our system requires for successful implementation.
        					    \item \textbf{Other}\\
        					    Other libraries and API's will then be added to the project later to cater for any extensions once our core features have been implemented.
        					\end{itemize}
                                              \subsubsection{Hardware Interfaces}
    						\begin{itemize}
    						    \item \textbf{Camera} \\
    						    We will use the Camera if a phone does not have a NFC chip to read a QR code that consist of the business card information . \\
						\end{itemize}
						\subsubsection{Communication Interfaces}
						\begin{itemize}
    						    
    						    \item \textbf{Near field communication (NFC)} \\
    						    We will make useNFC to write and read data from a NFC smart card.    						    
    						  
    						    \item \textbf{E-mail} \\
    						    We will use email for registration and login, which will then allow us to identify different users.  \\
    						\end{itemize}

 				\subsection{Functional Requirements}
				\subsection{Performance Requirements}
				\subsection{Design Constraints}
				\paragraph\indent
				This application is constrained in two main areas, namely hardware and software. We will look at these two areas separately.
				 \subsubsection{Hardware}
					\begin{enumerate}
						\item The SmartCard can only store a small amount of information since it has limited space
						
					\end{enumerate}
				
				 \subsubsection{Software}
					\begin{enumerate}
						\item The system must be able to operate efficiently.
                                                 \item The system must be able to keep all user information secure.
					\end{enumerate}
			
				 \subsubsection{Other}
				
					
									\newpage
				\subsection{Software System attributes}
				
				\subsubsection{Reliability}
    				The system must be able to perform all its functions under the stated conditions and produce correct and consistent results. 
    				
				
				\subsubsection{Security}
    					\begin{itemize}
					\item Only administrators must be able to write information to the SmartCard.
					\item All data must be transmitted in a securely.

					\end{itemize}
    				
				\subsubsection{Reusability}
	
				    
				\subsubsection{Efficiency}
				  The system must be able to perform various functions and produce desired results with minimum expenditure of time and resources.which is the sending of correct business card information and also store the correct information on the receiving device.
				\subsubsection{Availability}
	                          The system must be always available provided that the minimum hardware requirements are met.
				 
	
				\subsubsection{Interoperability}
				

				\subsection{Other Requirements}
				

	
\end{document}
