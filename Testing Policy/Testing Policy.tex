
\documentclass[english]{article}
\usepackage[utf8]{inputenc}
\usepackage[margin=1in]{geometry}
\usepackage{graphicx,wrapfig,lipsum}
\usepackage{caption}
\usepackage{refstyle}
\usepackage{titlepic}
\usepackage{graphicx}
\usepackage{grffile}
\usepackage{babel}
\usepackage{parskip}
\textwidth = 426pt
\oddsidemargin = 17pt


\title{Cos301 : Test Policy\\
	for the SmartCard Application\\
}
\date{\today}
\graphicspath{{Pictures/}}

\begin{document}
	\maketitle
	\begin{figure}[!t]
		\includegraphics{up_logo.png}
	\end{figure}
	\begin{minipage}{0.4\textwidth}
		\begin{flushleft} \large
			\textbf{NAMES:}\\[0.4cm]
			Mufamadi {Khodani} \\
			Setoaba {Phuti} \\
			Dandashe {Kagiso Xolo} \\
			Mashaba {Mpho} \\
			Mathe {Musa} \\
			Ntsoane {Thabo}\\
		\end{flushleft}
	\end{minipage}
	\begin{minipage}{0.4\textwidth}
		\begin{flushright} \large
			\textbf{STUDENT NUMBER:} \\[0.4cm]
			u14197520 \\
			u13032616 \\
			u14245681 \\
			u14309999 \\
			u15048030 \\
			u15107532 \\
		\end{flushright}
	\end{minipage}
	
	
	\pagenumbering{gobble}
	\newpage
	
	\tableofcontents
	
	
	\pagenumbering{arabic}
	
	\newpage

	


	\pagenumbering{arabic}
	

	\section{Mission of Testing}
	To effectively deliver efficient and effective software that meets the customers requirements. To further ensure that quality software is produced.  


	\section{Test Description}
	Our Testing process will be divided into two sections. Unit testing and manual testing. 
	\subsection {Unit testing}
			\subsubsection{Storage}
		\begin{itemize} 
			\item Check if the information is stored correctly.
			\item Check if the information can be retrieved from the database.
		\end{itemize}
			\subsubsection{Application activities and functions}
			\begin{itemize}
				\item Ensure that an appointment can be set and retrived
				\item Validate that emails are of the correct form 
				\item Ensure that a user can be created
			\end{itemize}
	
	\subsection {Manual testing}
	In manual testing the test cases will be excuted manually as the name suggests(by a human being that is). 
		\subsubsection{QR Code functionality}
		\begin{itemize} 
			\item Check if the QR code generates correct information
			\item Check the scanner scans properly
			\item Check that the information transmitted to another device is accurate. 
		\end{itemize}
		
		\subsubsection{NFC functionality}
		\begin{itemize} 
			\item Check if the App notifies the user if the NFC technology is not switched on.
			\item ensure that the device sends information
			\item ensure that the information transmitted is accurate
		\end{itemize}
		\subsubsection{Import Card}
		\begin{itemize}
			\item Ensure that a physical card can be scanned
			\item Ensure that the information that extracted is as accurate as possible.tent...
		\end{itemize}
		
			
	
		

	\section{Test Evaluation}
	
	We will use JUnit and  Travis-CI to ensure that implementations have been functions as it should.
	Travis-CI is a continuous integration, hosted service,  JUnit is a testing framework for Android and Java applications.
	\section{Quality level to be achieved}
	We aim to have no outstanding faults prior to product release.
	
	\section{Approach to Test process improvement}
	
	Project review meetings are held in regards to improving the project.
	
	 
	
	\section{Travis-CI yml file}
	language: android
	
	android:
	components:
	- platform-tools
	- tools
	
	
	//The SDK version we used to compile our project:
	 android-26
	
	//The BuildTools version used in our project: 
	 build-tools-26.0.2
	
	//Other additional components:  extra-google-google-play-services
	
	//The emulator that we will be using:
	 sys-img-armeabi-v7a-android-17	
	 
	\section{Testing Scripts}
	
	RegTest.java
		
\end{document}
